\documentclass{article}
\usepackage[T1]{fontenc}
\usepackage[utf8]{inputenc}

\usepackage{xspace}
\usepackage{graphicx}
\usepackage{amsmath}
\usepackage{amsfonts}
\usepackage{subfig}
\usepackage{fullpage}
\usepackage{url,paralist}

\author{Philip Pickering\\ \url{pgpick@gmx.at} \and Marco Eilers\\ \url{eilers.marco@googlemail.com} \and Thomas Bracht Laumann Jespersen\\ \url{ntl316@alumni.ku.dk}}
\title{Statistical Methods for Machine Learning\\ Assignment 2: Basic Learning Algorithms}
\date{}

\newcommand{\vect}[1]{\ensuremath{\boldsymbol{\mathbf{#1}}}\xspace}
%%\newcommand{\M}{\ensuremath{\begin{pmatrix}100 & 0 

%% %% Uncomment these lines to get a box around all figures
%% \usepackage{float}
%% \floatstyle{boxed}
%% \restylefloat{figure}

\begin{document}
\maketitle

\section{Regression}

\subsection{Maximum Likelihood solution}

Use linear model
\[
y(\vect{x},\vect{w}) = w_0 + w_1 x_1 + w_2 x_2 + \dots + w_D x_D
\]
and for the $D$ variables we let $\phi_i(\vect{x}) = x_i$ for
$i = 1,\dots,D$ and $\phi_0(\vect{x}) = 1$. 
\subsubsection{Selection 1}

For our first selection $S_1$ our design matrix becomes a $200\times
5$ matrix.

\[
\vect{\Phi}_{S_1} = \begin{bmatrix}1 & \vect{x}_{1,1} & \vect{x}_{1,2} & \vect{x}_{1,3} & \vect{x}_{1,4}\\
\hfill & \hfill & \vdots & \hfill & \hfill\\
1 & \vect{x}_{i,1} & \vect{x}_{i,2} & \vect{x}_{i,3} & \vect{x}_{i,4}\\
\hfill & \hfill & \vdots & \hfill & \hfill\\
1 & \vect{x}_{N,1} & \vect{x}_{N,2} & \vect{x}_{N,3} & \vect{x}_{N,4}\\
\end{bmatrix}
\]
where the notation $\vect{x}_{i,j}$ indicates the $j$'th entry in the
$i$'th vector. %%Similarly we obtain a design matrix for our second
%%selection, $S_2$ of dimensions $200\times 2$, with $D = 2$:

Finding the ML estimate of our parameters for $S_1$ gives
\[
\vect{w}_{S_1} = \left[\!\begin{tabular}{r@{.}l} -43 & 0947\\ -0 &
  1299\\ 0 & 0352\\ 0 & 9335\\ -0 &
  0433\end{tabular}\!\right]\quad\text{ and }\quad \mathrm{RMS}_{S_1} = 4.3897
\]

\subsubsection{Selection 2}

Our second selection $S_2$ consists only of the data from the `Abdomen
2' column, giving a design matrix $\vect{\Phi}_{S_2}$ of dimensions
$200\times 2$. Training the model on the same training data yields:
\[
\vect{w}_{S_2} = \left[\!\begin{tabular}{r@{.}l}
  -37 & 4085\\
  0 & 6133
  \end{tabular}\!\right]\quad
\text{ and }\quad \mathrm{RMS}_{S_2} = 5.2064
\]

%% Conclusion
\subsubsection{Discussion}
Just looking at the root mean square values of the two selections, it
appears that $S_1$ performs better than $S_2$, but not by a lot. This
suggests that either the variable `Abdomen 2' is the most descriptive
in terms of body, or that the linear model simply is a poor fit no
matter how many variables we include. It could be a combination of the
two.

It could probably be argued that the linear regression model is a poor
predictor, but including more variables should improve the results.

\subsection{Maximum a posteriori solution}

Use MAP estimation. Fix prior to zero mean isotropic Gaussian (eq. 3.52)
\[
p(\vect{w}|\alpha) = \mathcal{N}(\vect{w}|\vect{0},\alpha^{-1}\vect{I})
\]
Set noise precision parameter $\beta = 1$. Compute (3.53) and (3.54)
\[
\text{(3.53)}\quad \vect{m}_N = \beta\vect{S}_N\vect{\Phi}^T\mathsf{t}
\]
and
\[
\text{(3.54)}\quad \vect{S}_N^{-1} = \alpha\vect{I} + \beta\vect{\Phi}^T\vect{\Phi}
\]

Apply model to test set and compute RMS error for different values of the precision parameter $\alpha$.

\subsection{Theory}

Verify result in equation (3.49) for the posterior distribution of the
parameters \vect{w} in the linear basis function in which $\vect{m}_N$
and $\vect{S}_N$ are defined 

\section{Linear Discriminant Analysis}

Visualize the training data sets in three 2D plots.

Apply LDA to the training data, report accuracies of the classifier on
the training as well as on the test sets. Explain the results. Discuss
similarities and differences in performance on three data sets. Could
a non-linear method do better?

Implement LDA algorithm by hand.

\section{Nearest Neighbor Classification}

\subsection{Nearest Neighbor Classification with Euclidian Metric}

Implement $k$-NN. Train for all three \textsc{knoll} problems using
training data sets. Report accuracy on corresponding sets for $k =
1,2,3,\dots,9$. Explain results; discuss similarities and differences
in performance on the three data sets. Compare results to LDA results.

Hand in: classifier source code, results, short discussion.

\subsection{Changing the Metric}

To prove that $d$ is a metric, given
\[
d(\vect{x},\vect{z}) = \|\vect{M}\vect{x} - \vect{M}\vect{z}\|\text{,
  where } \vect{M} = \begin{pmatrix} 100 & 0 \\ 0 & 1\end{pmatrix}
\]
and $\|\cdot\|$ is the standard $L_2$-norm (in $\mathbb{R}^2$), we
need to verify $\forall \vect{x},\vect{y}\in \mathbb{R}^2$ that
\begin{inparaenum}[1)]
  \item $d(\vect{x},\vect{y}) \geq 0$; 
  \item $d(\vect{x},\vect{y}) = 0 \Leftrightarrow \vect{x} = \vect{y}$;
  \item $d(\vect{x},\vect{y}) =  d(\vect{y},\vect{x})$ (symmetry) and
  \item $\forall \vect{x},\vect{y},\vect{z} \in \mathbb{R}^2 : d(\vect{x},z) \leq d(\vect{x},\vect{y}) + d(\vect{y},\vect{z})$.
\end{inparaenum}

\subsubsection{Revised proof --- basic observations}

We need only to observe that $\vect{M}$ is a projection $m : \mathbb{R}^2 \rightarrow \mathbb{R}^2$, i.e.\ onto $\mathbb{R}^2$ itself, given by $m(\vect{x}) = \vect{M}\vect{x}$. This immediately gives us all the properties we need, because $L_2$ is itself a (complete) metric on $\mathbb{R}^2$.

For instance, if we let $\vect{x},\vect{y},\vect{z} \in \mathbb{R}^2$ and $\vect{x'},\vect{y'},\vect{z'}$ be the result of applying $m$ on $\vect{x},\vect{y},\vect{z}$ respectively, we can prove the triangle inequality:

\begin{align}
\nonumber d(\vect{x},\vect{z}) &= \left\|\vect{M}\vect{x} - \vect{M}\vect{z}\right\|\\
\nonumber &= \left\|\vect{M}\vect{x} - \vect{M}\vect{y} + \vect{M}\vect{y} - \vect{M}\vect{z}\right\|\\
\nonumber &= \left\|(\vect{x'} - \vect{y'}) + (\vect{y'} - \vect{z'})\right\|\\
\nonumber &\leq \left\|\vect{x'} - \vect{y'}\right\| + \left\|\vect{y'} - \vect{z'}\right\|\quad \text{(by property of } L_2 \text{ norm)}\\
  &= d(\vect{x},\vect{y}) + d(\vect{y},\vect{z})
\end{align}

\subsubsection{Original proof --- the long way}

Let $\vect{x}, \vect{y} \in \mathbb{R}^2$. As it is easier to consider
the square of the $L_2$ norm, we will do so:
\begin{align}
\nonumber  d(\vect{x},\vect{y})^2 &= \left\|\begin{pmatrix} 100 & 0\\ 0 & 1\end{pmatrix}\begin{pmatrix}x_1\\ x_2\end{pmatrix} - \begin{pmatrix}100 & 0\\ 0 & 1\end{pmatrix}\begin{pmatrix}y_1\\ y_2\end{pmatrix}\right\|^2 \\
      \nonumber &= \left\| \begin{pmatrix} 100(x_1 - y_1)\\ x_2 - y_2\end{pmatrix}\right\|^2\\
\nonumber          &= \begin{pmatrix} 100(x_1 - y_1)\\ x_2 - y_2\end{pmatrix}^T\begin{pmatrix} 100(x_1 - y_1)\\ x_2 - y_2\end{pmatrix}\\
            &= (100(x_1 - y_1))^2 + (x_2 - y_2)^2 \label{eq:c1}.
\end{align}

From \eqref{eq:c1} we observe that $(x_1 - y_1)^2 \geq 0$ and
similarly $(x_2 - y_2)^2 \geq 0$ for all values in $\mathbb{R}$, so
our first criteria for a metric is fulfilled. We also observe that if
$d(\vect{x},\vect{y}) = 0$ it implies $x_1 - y_1 = 0$ and $x_2 - y_2 =
0$, which means that $x_1 = y_1$ and $x_2 = y_2$, thus $d$ fulfills
our second requirement for a metric.

Our requirement of symmetry requires a little more
investigation. Proceeding from \eqref{eq:c1}, we find
\begin{align}
\nonumber (100(x_1 - y_1))^2 + (x_2 - y_2)^2 &= 100^2(x_1^2 - 2x_1y_1 + y_1^2) + (x_2^2 - 2x_2y_2 + y_2^2)\\
  \nonumber &= 100^2(y_1^2 - 2y_1x_1 + x_1^2) + (y_2^2 - 2y_2x_2 + x_2^2)\\
  \nonumber &= (100(y_1 - x_1))^2 + (y_2 - x_2)^2\\
  &= d(\vect{y},\vect{x})^2\label{eq:c2},
\end{align}
where \eqref{eq:c2} gives us $d(\vect{x},\vect{y}) =
d(\vect{y},\vect{x})$ and therefore fulfills the symmetry
requirement.

Our last requirement is the triangle inequality, i.e.\ $d(\vect{x},\vect{z}) \leq d(\vect{x},\vect{y}) + d(\vect{y},\vect{z}), \forall \vect{x},\vect{y},\vect{z}\in \mathbb{R}^2$.

Let $\vect{x},\vect{y},\vect{z}\in\mathbb{R}^2$.
\begin{align}
  d(\vect{x},\vect{z}) &= \left\|\vect{M}\vect{x} - \vect{M}\vect{z}\right\|\\
  &= \left\|\vect{M}\vect{z} - \vect{M}\vect{y} + \vect{M}\vect{y} - \vect{M}\vect{z}\right\|\\
  &\leq \left\|\vect{M}\vect{z} - \vect{M}\vect{y}\right\| + \left\|\vect{M}\vect{y} - \vect{M}\vect{z}\right\|\\
  &= d(\vect{x},\vect{y}) + d(\vect{y},\vect{z})
  %% &= (\vect{M}\vect{x} - \vect{M}\vect{z})^T(\vect{M}\vect{x} - \vect{M}\vect{z})\\
  %% &= ((\vect{M}\vect{x})^T - (\vect{M}\vect{z})^T)(\vect{M}\vect{x} - \vect{M}\vect{z})\\
  %% &= (\vect{M}\vect{x})^T\vect{M}\vect{x} \\
  %% &= \quad\vdots\quad \text{(We want to arrive at the following)} \\
  %% &\leq \left\|\vect{M}\vect{x} - \vect{M}\vect{y}\right\|^2 + \left\|\vect{M}\vect{y} - \vect{M}\vect{z}\right\|^2\\
  %% &= d(\vect{x},\vect{y})^2 + d(\vect{y},\vect{z})^2
\end{align}

\subsubsection{Results}
Use $d$ as metric in $k$-NN classifier and apply to
\textsc{knollC}. Explain results.


\end{document}
